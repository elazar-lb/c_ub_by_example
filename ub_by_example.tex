\documentclass{beamer}
\usetheme{Hannover}

\usepackage[utf8x]{inputenc}
\usepackage{minted}

\title{Undefined Behavior}
\subtitle{Through real world examples}
\author{Elazar Leibovich}
\institute{Lightbits Labs}
\date{\today}

\begin{document}

\begin{frame}
\titlepage
\end{frame}

\section Integer Overflow

\defverbatim[colored]\lstI{
\begin{minted}{C}
int foo (int x) {
 return (x + 1) > x;
}
int main () {
 printf("%d\n", (INT_MAX + 1) > INT_MAX);
 printf("%d\n", foo(INT_MAX));
 return 0;
} 
\end{minted}
}

\begin{frame}{What does it print?}
\lstI
\end{frame}

\begin{frame}{UB is unstable}{UB can display incoherent behavior}
\lstI
\begin{block}{Output}
gcc -O2 intmax-overflow.c ; ./a.out\\
0\\
1
\end{block}
\end{frame}

\begin{frame}{UB is unstable}{UB can display incoherent behavior}
\end{frame}

\section Division by Zero

\defverbatim[colored]\lstII{
\begin{minted}{C}
void f(int msize) {
  if (!msize)
    msize = 1 / msize; /* Provoke a signal. */
}
\end{minted}
}

\defverbatim[colored]\lstIIAsm{
\begin{minted}{asm}
; gcc -S -fno-asynchronous-unwind-tables -o- x.c
f:
  ret
\end{minted}
}

\begin{frame}{UB can delete your code}{happened to a real project}
\lstII
Source: \hyperlink{https://lists.gnupg.org/pipermail/gcrypt-devel/2012-July/001958.html}{gcrypt-devel}
\end{frame}

\begin{frame}{UB can delete your code}{happened in gcrypt}
\lstII
\lstIIAsm
\end{frame}

\begin{frame}{UB can delete your code}{happened in gcrypt}
  \begin{itemize}
    \item Compiler see division by 0
    \item It's UB, hence can't happen
    \item Optimize away to no-op
  \end{itemize}
\end{frame}

\section Division Overflow

\defverbatim[colored]\lstIII{
\begin{minted}{C}
void f(int arg1, int arg2) {
  int result = arg1 / arg2;
  // Overflow check...
  if (arg2 == -1 && arg1 < 0 && result <= 0)
    ereport(ERROR,
      (errcode(ERRCODE_NUMERIC_VALUE_OUT_OF_RANGE),
      errmsg("bigint out of range")));
}
\end{minted}
}

\begin{frame}{Primary school math is tricky}{happened in Postgres}
  \lstIII
Where is the problem?
\end{frame}

\defverbatim[colored]\lstIIIrev{
\begin{minted}{C}
void f(int arg1, int arg2) {
  assert(arg2 != 0);
  int result = arg1 / arg2;
  // Overflow check...
  if (arg2 == -1 && arg1 < 0 && result <= 0)
    ereport(ERROR,
      (errcode(ERRCODE_NUMERIC_VALUE_OUT_OF_RANGE),
      errmsg("bigint out of range")));
}
\end{minted}
}

\begin{frame}{Primary school math is tricky}{happened in Postgres}
  \lstIIIrev
Better now?
\end{frame}

\defverbatim[colored]\lstIIIasm{
\begin{minted}{C}
int result = arg1 / arg2;
\end{minted}
\begin{minted}{asm}
 	mov    %edi,%eax
 	cltd   ; sign extend eax to edx:eax
 	idiv   %esi ; eax = edx:eax / esi
\end{minted}
What does idiv do?
}

\begin{frame}{Division is tricky}{happened in Postgres}
  \lstIIIasm
\end{frame}

\begin{frame}{Division is tricky}{happened in Postgres}
  \begin{itemize}
    \item Signed 8-bit integer range: $-128-127$
    \item What's the result of $-128/-1$
    \item UB
    \item Postgres assumed it'll be rounded below 0
  \end{itemize}
\end{frame}

\defverbatim\listIIIdivalg{
\begin{minted}{C}
int result = arg1 / arg2;
temp ← AX / SRC; (* Signed division *)
  IF (temp > 7FH) or (temp < 80H)
  (* If a positive result is greater than 7FH or a negative result is less than 80H *)
    THEN #DE; (* Divide error *)
\end{minted}
}

\begin{frame}{Division is tricky}{happened in Postgres}
  IDIV — Signed Divide\\
temp ← AX / SRC; (* Signed division *) \\
  IF ($temp > 7FH$) or ($temp < 80H$) \\
  (* If a positive result is greater than 7FH or a negative result is less than 80H *) \\
    THEN \#DE; (* Divide error *)\\\\

  Source: \href{https://www.felixcloutier.com/x86/idiv}{Intel SDM}
\end{frame}

\section Condition null and void

\begin{frame}{UB affects unrelated code}{happened in kernel}

  Source: \hyperlink{LWN}{https://lwn.net/Articles/342330/}
\end{frame}

\section NaCI

\defverbatim\listVI{
\begin{minted}{C}
return addr & ~(uintptr_t)((1 << nap->align_boundary) - 1);
\end{minted}
}

\begin{frame}{Even competent programmers does UB}{NacI case}

  Source: \hyperlink{NaCI \#245}{http://code.google.com/p/nativeclient/issues/detail?id=245}
\end{frame}


\end{document}
